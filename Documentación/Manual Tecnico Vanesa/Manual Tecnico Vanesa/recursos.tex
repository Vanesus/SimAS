
\chapter{Recursos}

En este cap�tulo se expondr�n de forma clara y concisa los recursos humanos y materiales necesarios para este proyecto. Los recursos se definen como aquellos medios de los que se dispone para abordar el proceso de desarrollo del proyecto.\\

Los recursos se encuentran agrupados en tres grupos:
\begin{itemize}

\item Recursos humanos.
\item Recursos de hardware.
\item Recursos de software.

\end{itemize}

\section{Recursos humanos}

\begin{itemize}

\item Direcci�n y coordinaci�n: Dr. Nicol�s Luis Fern�ndez Garc�a. Profesor Titular de Universidad del �rea de conocimiento de Ciencias de la Computaci�n e Inteligencia Artificial. Departamento de Inform�tica y An�lisis Num�rico, Escuela Polit�cnica Superior de C�rdoba, Universidad de C�rdoba.

\item An�lisis, dise�o, programaci�n y documentaci�n: Vanesa Gonz�lez P�rez. Alumna de la titulaci�n Ingenier�a Inform�tica, Escuela Polit�cnica Superior de C�rdoba, Universidad de C�rdoba. 

\end{itemize}

\section{Recursos de Hardware}

Los recursos hardware se pueden subdividir en dos categor�as:

\begin{itemize}

\item Recursos de hardware para el desarrollo del proyecto: necesarios para llevar a cabo el desarrollo de la aplicaci�n. 

	\begin{itemize}
	
	\item Ordenador port�til Asus PRO5IJSeries. Intel Core i3-370M, 2,4 GHz. 4 GB de memoria RAM. HDD de 500 GB. Tarjeta gr�fica Radeon HD 6370m.	
	
	\end{itemize}
	
\item Recursos hardware para el uso de la aplicaci�n: requisitos m�nimos recomendados para el buen funcionamiento de la aplicaci�n.

	\begin{itemize}
	\item Requisitos m�nimos declarados por Sun Microsystems para que la M�quina Virtual de Java pueda funcionar correctamente.
	\end{itemize}
	
\end{itemize}		
	

\section{Recursos de Software}

\begin{itemize}

\item Sistemas operativos:
	\begin{itemize}
	\item Ubuntu Linux, versi�n 12.10
	\item Microsoft Windows 7
	\end{itemize}	 

\item Int�rpretes: Java SE (JDK) 7u40
\item Editores: editor de textos Texmaker para Latex
\item Programaci�n y planificaci�n del proyecto: gestor de proyectos OpenProj, para gestionar la organizaci�n temporal y la planificaci�n.
\item Entornos de desarrollo. NetBeans 7.3.1. Se utilizar� para la codificaci�n y pruebas de la aplicaci�n.
\item Herramientas de diagramado: se utilizar� el editor de diagramas Dia para desarrollar todos los diagramas del proyecto (diagramas de casos de uso, de clases, etc.).

\end{itemize}