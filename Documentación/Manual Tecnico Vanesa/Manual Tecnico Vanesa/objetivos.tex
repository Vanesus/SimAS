
\chapter{Objetivos}

De acuerdo a la identificaci�n real y t�cnica del problema que se ha realizado en el cap�tulo anterior, a continuaci�n se expondr�n todos los objetivos funcionales que se pretenden alcanzar con el desarrollo de este proyecto.

\section{Objetivo principal}

El principal objetivo es el de construir una herramienta completa que facilite el aprendizaje del An�lisis  Sint�ctico tanto ascendente como descendente, de forma que permita a todas las personas interesadas en la materia aprender de una manera sencilla, r�pida y amena.

El software deber� permitir realizar las siguientes tareas principales:

\begin{enumerate}

\item Editor de gram�ticas de contexto libre.

\item An�lisis descendente y predictivo.

\item An�lisis ascendente:

\begin{itemize}

\item An�lisis SLR.

\item An�lisis LR-can�nico.

\item An�lisis LALR.

\end{itemize}

\item Detecci�n y recuperaci�n de errores.

\item Tutorial.

\item Ayuda.

\end{enumerate}

\section{Objetivos espec�ficos}

Adem�s de las tareas principales comentadas anteriormente, la aplicaci�n deber� cumplir los siguientes objetivos secundarios:

\begin{itemize}
  \item Editor de gram�ticas:
    \begin{itemize}
      \item Creaci�n de una gram�tica de contexto libre.
      \item Edici�n del vocabulario.
      \item Gesti�n de producciones.
      \item Validaci�n de la gram�tica.
      \item Carga y almacenamiento de la gram�tica.
      \item Generaci�n de informes.
    \end{itemize}
  \item Simulador descendente:
    \begin{itemize}
     \item Construcci�n de la tabla predictiva.
     \item Simulaci�n de la gram�tica.
     \item Gesti�n de errores.
     \item Generaci�n de informes.
    \end{itemize}
  \item Simulador ascendente:
  	\begin{itemize}
  	 \item Construcci�n de la tabla de an�lisis LR.
  	 \item Simulaci�n de los m�todos: SLR, LR-can�nico y LALR.
	 \item Gesti�n de errores.
	 \item Generaci�n de informes.  	 
  	\end{itemize}
  \item Tutorial
    \begin{itemize}
     \item Conceptos te�ricos del an�lisis sint�ctico.
     \item Ejemplos.
    \end{itemize}
  \item Ayuda.
    \begin{itemize}
     \item Instalaci�n y desinstalaci�n.
     \item Interfaz y modulos del programa.
     \item Resoluci�n de problemas.
    \end{itemize}

\end{itemize}






