%==================================================================================
%                              BIBLIOGRAF�A
%==================================================================================

%\pagestyle{myheadings} \markboth{\small{Simulador Gr�fico de An�lisis Sint�cticos - Referencias}}{\small{Referencias}}

% Se escribe la bibliograf�a.
%\begin{thebibliography}{0}

\vspace*{15cm}
\clearpage
\thispagestyle{empty}
\vspace*{2cm}

{\Huge{\textbf{Referencias}}}
\\


%  \bibitem{1} 
\begin{enumerate}
	\item Descarga de \textbf{Netbeans}: \\ \textbf{�ltimo acceso}: junio de 2015. \\ \textbf{URL}: {\color{blue}\underline{http://netbeans.org/downloads}}\\
	
	\item Descarga de \textbf{Java}: \\ \textbf{�ltimo acceso}: junio de 2015. \\ \textbf{URL}: {\color{blue}\underline{https://www.java.com/es/download/}}\\
	
	\item Tutorial de \textbf{Java Swing}: \\ \textbf{�ltimo acceso}: junio de 2015. \\ \textbf{URL}: {\color{blue}\underline{http://docs.oracle.com/javase/tutorial/uiswing/}}\\
	
	\item  \textbf{Itext}: \\ \textbf{�ltimo acceso}: junio de 2015. \\ \textbf{URL}: {\color{blue}\underline{http://itextpdf.com/}}\\
	
	\item  \textbf{XML}: \\ \textbf{�ltimo acceso}: junio de 2015. \\ \textbf{URL}: {\color{blue}\underline{http://www.w3.org/XML/}}
	
	
\end{enumerate}
  



 
%\end{thebibliography}

% Se a�ade la bibliograf�a al �ndice de contenidos.
\addcontentsline{toc}{part}{\hspace{0.4cm} Referencias} 

% Se deja una p�gina en blanco.
\newpage{\pagestyle{empty}\cleardoublepage} 