

\usepackage[latin1]{inputenc}
\usepackage[spanish]{babel}
\usepackage{amsmath}
\usepackage{amsfonts}
\usepackage{amssymb}
\usepackage{makeidx}
\usepackage{graphicx}
\usepackage[left=2cm,right=2cm,top=2cm,bottom=2cm]{geometry}
\usepackage{float}
\usepackage{colortbl}	% Coloreado de columnas de tablas
\usepackage[cmyk]{xcolor}
\usepackage{multirow}
\usepackage{rotating}
\usepackage{longtable}
\usepackage{pdfpages}
\usepackage[latin1]{inputenc}
\usepackage{listings}

\usepackage{color}
\definecolor{gray}{rgb}{0.4,0.4,0.4}
\definecolor{darkblue}{rgb}{0.0,0.0,0.6}
\definecolor{cyan}{rgb}{0.0,0.6,0.6}

%\lstset{
%  basicstyle=\ttfamily,
%  columns=fullflexible,
%  showstringspaces=false,
%  commentstyle=\color{gray}\upshape
%}

\lstset{ %
language=Java,                % choose the language of the code
basicstyle=\footnotesize,       % the size of the fonts that are used for the code
numbers=left,                   % where to put the line-numbers
numberstyle=\footnotesize,      % the size of the fonts that are used for the line-numbers
stepnumber=1,                   % the step between two line-numbers. If it is 1 each line will be numbered
numbersep=5pt,                  % how far the line-numbers are from the code
backgroundcolor=\color{white},  % choose the background color. You must add \usepackage{color}
showspaces=false,               % show spaces adding particular underscores
showstringspaces=false,         % underline spaces within strings
showtabs=false,                 % show tabs within strings adding particular underscores
frame=single,           % adds a frame around the code
tabsize=2,          % sets default tabsize to 2 spaces
captionpos=b,           % sets the caption-position to bottom
breaklines=true,        % sets automatic line breaking
breakatwhitespace=false,    % sets if automatic breaks should only happen at whitespace
escapeinside={\%*}{*)}          % if you want to add a comment within your code
}

\lstdefinelanguage{XML}
{
  morestring=[b]",
  morestring=[s]{>}{<},
  morecomment=[s]{<?}{?>},
  stringstyle=\color{black},
  identifierstyle=\color{darkblue},
  keywordstyle=\color{cyan},
  morekeywords={xmlns,version,type}% list your attributes here
}

\author{Vanesa Gonz�lez P�rez}

\newcommand{\HRule}{\rule{\linewidth}{0.5mm}}
\renewcommand{\arraystretch}{1.5} % alto de la tabla


\newcommand{\Dflecha}{\mathop{\Rightarrow}\limits}
\newcommand{\ldflecha}{\mathop{\longrightarrow}\limits}


%CONTROLA LOS MARGENES DEL TEXTO
% margen {izquierda}{derecha}{arriba}{abajo}
\usepackage{anysize}
\marginsize{3cm}{3cm}{2.5cm}{2.5cm}

%\headheight = 11pt
%\topmargin=-0.68cm % Controla la altura
\headsep=0.7cm % margen de separación entre la cabecera y el texto

\setlength{\parindent}{7mm}%sangria la primera linea

%Para conseguir que en las páginas en blanco no ponga cabeceras
\makeatletter
\def\clearpage{%
  \ifvmode
    \ifnum \@dbltopnum =\m@ne
      \ifdim \pagetotal <\topskip
        \hbox{}
      \fi
    \fi
  \fi
  \newpage
  \thispagestyle{empty}
  \write\m@ne{}
  \vbox{}
  \penalty -\@Mi
}

%  FORMATO ENCABEZADO
%==========================================
\usepackage{fancyhdr}
\pagestyle{fancy}
\fancyhf{} %borra los ajustes del encabezado
\fancyhead[LO]{\leftmark} % En las páginas impares, parte izquierda del encabezado, aparecerá el nombre de capítulo
\fancyhead[RE]{\rightmark} % En las páginas pares, parte derecha del encabezado, aparecerá el nombre de sección
\fancyhead[RO,LE]{\thepage} % Números de página en las esquinas de los encabezados

%  NOTAS AL PIE DE PÁGINA
%==========================================

\footnotesep=0pt  %Separación de las líneas en las notas a pie de página
\newcommand{\newfootnote}[1]{\footnote{\hspace*{2mm}#1}}%Nueva nota de pie de página.
\renewcommand{\baselinestretch}{1}%Ancho de separación entre líneas dentro de un párrafo
\footskip = 30cm
\usepackage[marginal,flushmargin,norule]{footmisc} %paquete para controlar las notas de pie de página
\setlength\footnotemargin{1.8em} %Margen de la nota de pie de página
\setlength{\skip\footins}{8mm}%Separación entre el texto principal y las notas pie de pagina
\setlength{\parskip}{15pt}

%DEFINICIÓN DE ÓRDENES
\def\href#1#2{ #1 (#2)}
\def\url#1{#1}
\def\xmlsch{\textit{XML Schema }}
\def\xmlschf{\textit{XML Schema}}
\def\latex{\textit{Latex}}
\def\tex{\textit{Tex}}
\def\winedt{\textit{WinEdt}}

% Profundidad de la tabla de contenido.
\setcounter{tocdepth}{3} % nivel de sección y subsección, etc.
\setcounter{secnumdepth}{3}

% Se declaran las extensiones de los gr�ficos. Por defecto, .eps.
\DeclareGraphicsExtensions{.eps}

% Declaracion de colores
\definecolor{verdeOscuro}{rgb}{0,0.5,0}
\definecolor{naranjaOscuro}{rgb}{0.5,0.25,0}


\colorlet{verde-oscuro}{green!20!black}
\colorlet{verde-oscuro1}{green!50!green}
\colorlet{verde}{green!50!black}

\colorlet{marron-oscuro}{brown!20!black}
\colorlet{marron}{brown!50!black}

\colorlet{amarillo-claro}{yellow!50!white}

\colorlet{rosa}{red!50!white}

